%%%%%%%%%%%%%%%%%%%%%%%%%%%%%%%%%%%%%%%%%
% Medium Length Professional CV
% LaTeX Template
% Version 2.0 (8/5/13)
%
% This template has been downloaded from:
% http://www.LaTeXTemplates.com
%
% Original author:
% Thanks : Rishi Shah 's Contribution
% inspired by his awesome contribution:
% https://www.overleaf.com/articles/rishi-shahs-resume/vgxvkmxktyxn
% Author : Allianzcortex
% contact me : github.com/Allianzcortex
% email : iamwanghz#gmail.com
%
% Important note:
% This template requires the resume.cls file to be in the same directory as the
% .tex file. The resume.cls file provides the resume style used for structuring the
% document.
%
%%%%%%%%%%%%%%%%%%%%%%%%%%%%%%%%%%%%%%%%%

%----------------------------------------------------------------------------------------
%	PACKAGES AND OTHER DOCUMENT CONFIGURATIONS
%----------------------------------------------------------------------------------------

\documentclass{resume} % Use the custom resume.cls style

\usepackage[left=0.40in,top=0.2in,right=0.4in,bottom=0.1in]{geometry} % Document margins
\usepackage{fontawesome}
\usepackage{times}
\usepackage[T1]{fontenc}

\usepackage[breaklinks=true]{hyperref}
\urlstyle{same}
\hypersetup{
    colorlinks=true,
    linkcolor=blue,
    filecolor=magenta,      
    urlcolor=blue,
}

\newcommand{\tab}[1]{\hspace{.2667\textwidth}\rlap{#1}}
\newcommand{\itab}[1]{\hspace{0em}\rlap{#1}}

\name{Alexander Ding}
\address{\faGithub{ \href{https://www.github.com/alexander-ding}{github.com/alexander-ding}} \faLink{ \href{https://alexander-ding.github.io}{alexander-ding.github.io}} \faEnvelope{  \href{mailto:ding@brown.edu}{ding@brown.edu}}}
\address{\faMobilePhone{  (617)-455-7815} \faMapMarker{ Boston, MA, USA}} % Your address

\begin{document}

\begin{rSection}{Education}

{\bf Brown University} {\small (Enrolled, pursuing a CS degree)} \hfill {\em September 2020 - Present} 

{\bf Commonwealth School} {\small (GPA 4.98/5.00)}  \hfill {\em September 2016 - May 2020} 
\\$\bullet$ \textbf{Programming Coursework}: Algorithms \& Data Structures, Computer Architecture, OS, Machine Learning
\\$\bullet$ \textbf{Math Coursework}: Linear Algebra, Mathematical Logic, Multivariable Calculus, Theoretical Calculus

\end{rSection}

\begin{rSection}{Achievements}
\begin{tabular}{@{}l@{}l@{}}
    {\bf Papers}:~ & {\bf A. Ding}, Q. Chen, Y. Cao and B. Liu, \href{https://www.researchgate.net/publication/340452614\_Retinopathy\_of\_Prematurity\_Stage\_Diagnosis\_Using\_Object\_Segmentation\_and\_Convolutional\_Neural\_Networks}{"Retinopathy of Prematurity Stage Diagnosis Using Object Segmentation and } \\
    & \href{https://www.researchgate.net/publication/340452614\_Retinopathy\_of\_Prematurity\_Stage\_Diagnosis\_Using\_Object\_Segmentation\_and\_Convolutional\_Neural\_Networks}{Convolutional Neural Networks,"} \textit{2020 International Joint Conference on Neural Networks (IJCNN)}, in press \\
    {\bf Technical Reports}:~ & \href{https://math.mit.edu/research/highschool/primes/materials/2019/Ding.pdf}{"An Evaluation of UPC++ by Porting Shared-Memory Parallel Graph Algorithms"} \\
    {\bf Awards}:~ & National Merit Finalist, Presidential Scholar Semi-Finalist {\small (pending Finalist decision)} \\
    {\bf Others}:~ & CS Club Founder, Math Team Co-Captain, NEC Symphony Orchestra (Cello), Fencing Varsity Sabre Captain
\end{tabular}
\end{rSection}

\begin{rSection}{Work Experience}
{\bf Machine Learning Researcher, University of Massachusetts Lowell} \hfill {\em June 2019 - Present}
\\{\textit{Researcher under the mentorship of Dr. Benyuan Liu}}
\\$\bullet$ Implemented an energy-efficient neural network using quantized MobileNet to recognize types of vegetables on Android devices
\begin{tabular}{@{}l@{}l@{}}
    $\bullet$~ & Developed a novel neural network pipeline that combines object segmentation and image classification to automate the diagnosis of \\
    & Retinopathy of Prematurity, achieving $13\%$ accuracy increase compared to previous architectures \\
\end{tabular} 
\\$\bullet$ First-authored paper and accepted by IJCNN 2020 for publication {\small (see \textbf{Achievements})}
\\$\bullet$ \underline{Utilized} Python, Kotlin, OpenCV, Machine Learning, NumPy,  Tensorflow, Jupyter Notebook, and LaTeX

{\bf CS Researcher, Massachusetts Institute of Technology} \hfill {\em January 2019 - January 2020} 
\\{\textit{MIT PRIMES (highly selective year-long research program)}}
\\
\begin{tabular}{@{}l@{}l@{}}
    $\bullet$~ & Investigated the scalability and robustness of UPC++, a distributed programming C++ library, by implementing a suite of parallel \\ 
    & graph algorithms and benchmarking its performance on the NERSC supercomputer \\
\end{tabular} 
\\$\bullet$ Compared UPC++'s performance with OpenMP on an AWS machine
\\$\bullet$ Authored technical report and presented on Fall PRIMES Conference 2019 {\small (see \textbf{Achievements})}
\\$\bullet$ \underline{Utilized} C++, Python, Parallel Algorithms, High Performance Computing, OpenMP, and LaTeX

{\bf Research Assistant, Massachusetts Institute of Technology} \hfill {\em September 2017 - April 2018} 
\\{\textit{Intern for Dr. Tobias Gerstenberg}}
\\$\bullet$ Created a web-based interface (with a physics engine) to simulate causality experiments using Box2D.js
\\$\bullet$ Incorporated a SQL backend to store experiment results
\\$\bullet$ \underline{Utilized} JavaScript, HTML/CSS, and MySQL

\end{rSection}

\begin{rSection}{Projects}
{\bf Personal Website:} {\small \href{https://alexander-ding.github.io}{\textit{https://alexander-ding.github.io}} (for additional information and projects)}

{\bf Neural Net Flowchart} {\small (\href{https://alexander-ding.github.io/nn-flowchart}{\textit{https://alexander-ding.github.io/nn-flowchart}})}
\\$\bullet$ Created a website to rapidly experiment, evaluate, and save neural network architectures using an intuitive GUI
\\$\bullet$ Designed an easy drag-and-drop interface using React.js
\\$\bullet$ Implemented a RESTful backend server to allow persistent model storage and link sharing
\\$\bullet$ \underline{Utilized}: Python, Flask, Heroku, PostgresSQL, React.js, Tensorflow.js, HTML/CSS, Docker, GIT

{\bf YeetBot} {\small (\href{https://top.gg/bot/563019457367375882}{\textit{https://top.gg/bot/563019457367375882}})}
\\$\bullet$ Built a Discord Bot (in >100 servers) using dlib that allows users to meme-ify images by overlaying identified faces with custom masks
\\$\bullet$ Incorporated OpenCV to support easy mask editing, as well as persistent user settings using a cloud-hosted server
\\$\bullet$ \underline{Utilized}: Python, dlib, Machine Learning, OpenCV, Heroku, Docker, GIT

{\bf Python Like You Mean It (Chinese version)} {\small (\href{https://cn.pythonlikeyoumeanit.com}{\textit{https://cn.pythonlikeyoumeanit.com}})}
\\$\bullet$ Created a Chinese version of PythonLikeYouMeanIt, a free online resource for learning the basics of Python and NumPy
\\$\bullet$ Hosted the translation online to be accessible to the Chinese programming community, in collaboration with original author
\\$\bullet$ \underline{Utilized}: NumPy, Markdown, Sphinx, GIT

{\bf Quote of the Week} {\small (\href{https://qotw.net/}{\textit{https://qotw.net}})}
\\$\bullet$ Utilized Google Firebase to implement an online site for high school's Quote of the Week
\\$\bullet$ \underline{Utilized}: Firebase, Bootstrap, React.js, Redux, GIT

\end{rSection}

\begin{rSection}{Skills}
\begin{tabular}{@{}l@{}l@{}}
    {\bf Software}:~ & (\textit{proficient}): Python, C++, Unix, GIT, SQL, LaTeX, Markdown, JavaScript (\textit{familiar}): C, Go, React.js, HTML/CSS, Docker \\
    {\bf Library}:~ & TensorFlow, NumPy, Flask, OpenMP, UPC++, Sphinx, Firebase
\end{tabular}
\end{rSection}


\end{document}